% 
% Annual Cognitive Science Conference
% Sample LaTeX Paper -- Proceedings Format
% 

% Original : Ashwin Ram (ashwin@cc.gatech.edu)       04/01/1994
% Modified : Johanna Moore (jmoore@cs.pitt.edu)      03/17/1995
% Modified : David Noelle (noelle@ucsd.edu)          03/15/1996
% Modified : Pat Langley (langley@cs.stanford.edu)   01/26/1997
% Latex2e corrections by Ramin Charles Nakisa        01/28/1997 
% Modified : Tina Eliassi-Rad (eliassi@cs.wisc.edu)  01/31/1998
% Modified : Trisha Yannuzzi (trisha@ircs.upenn.edu) 12/28/1999 (in process)
% Modified : Mary Ellen Foster (M.E.Foster@ed.ac.uk) 12/11/2000
% Modified : Ken Forbus                              01/23/2004
% Modified : Eli M. Silk (esilk@pitt.edu)            05/24/2005
% Modified : Niels Taatgen (taatgen@cmu.edu)         10/24/2006
% Modified : David Noelle (dnoelle@ucmerced.edu)     11/19/2014
% Modified : Roger Levy (rplevy@mit.edu)     12/31/2018



%% Change "letterpaper" in the following line to "a4paper" if you must.

\documentclass[10pt,letterpaper]{article}

\usepackage{cogsci}

%\cogscifinalcopy % Uncomment this line for the final submission 


\usepackage{pslatex}
\usepackage{apacite}
\usepackage{float} % Roger Levy added this and changed figure/table
                   % placement to [H] for conformity to Word template,
                   % though floating tables and figures to top is
                   % still generally recommended!

%\usepackage[none]{hyphenat} % Sometimes it can be useful to turn off
%hyphenation for purposes such as spell checking of the resulting
%PDF.  Uncomment this block to turn off hyphenation.


%\setlength\titlebox{4.5cm}
% You can expand the titlebox if you need extra space
% to show all the authors. Please do not make the titlebox
% smaller than 4.5cm (the original size).
%%If you do, we reserve the right to require you to change it back in
%%the camera-ready version, which could interfere with the timely
%%appearance of your paper in the Proceedings.



\title{How to Make a Proceedings Paper Submission}
 
\author{{\large \bf Morton Ann Gernsbacher (MAG@Macc.Wisc.Edu)} \\
  Department of Psychology, 1202 W. Johnson Street \\
  Madison, WI 53706 USA
  \AND {\large \bf Sharon J.~Derry (SDJ@Macc.Wisc.Edu)} \\
  Department of Educational Psychology, 1025 W. Johnson Street \\
  Madison, WI 53706 USA}


\begin{document}

\maketitle


\begin{abstract}
Include no author information in the initial submission, to facilitate
blind review.  The abstract should be one paragraph, indented 1/8~inch on both sides,
in 9~point font with single spacing. The heading ``{\bf Abstract}''
should be 10~point, bold, centered, with one line of space below
it. This one-paragraph abstract section is required only for standard
six page proceedings papers. Following the abstract should be a blank
line, followed by the header ``{\bf Keywords:}'' and a list of
descriptive keywords separated by semicolons, all in 9~point font, as
shown below.

\textbf{Keywords:} 
add your choice of indexing terms or keywords; kindly use a
semicolon; between each term
\end{abstract}

\section{Methods}

\subsection{Word Embeddings}

The study utilized 250-dimensional word embeddings, trained using the word2vec algorithm operationalized by the gensim package for Python 3. They were trained using gensim’s default parameters. Notably the window size for training was 5, each word needed to be more than one letter long, and each word needed to appear at least 5 times to be included in the model.

\subsection{Datasets}

In this study we used six datasets, four of which came from Reddit, one from CNN, and one from the Daily Mail. These datasets represent different contexts and topics which vary in terms of their formality. For a casual context we used Reddit, a social media site where anyone can post topics of discussion and have people comment their replies. The Reddit datasets we used is made up of comments rather than posts, as comments are more discussion oriented, whereas posts are often short descriptions or stories. The news datasets, which represented a formal context, are CNN and the Daily Mail.

We have also separated our datasets based on formality of topic being discussed. Two of our Reddit datasets pulled comments from the entire site, and thus mostly comprised of casual topics. Two of our Reddit datasets only used comments on posts concerning news and politics, thus being discussion on more formal or serious topics. Specifically, these formal topics were from comments on posts found in the communities r/politics, which focuses on American politics, r/news, which focuses on American news, and r/worldnews, which focuses on news from around the world. For the news datasets, the CNN was considered to be discussing formal topics and the Daily Mail was considered to be discussing casual topics. The reason for this is that CNN typically focuses more on traditional news and politics, whereas the Daily Mail is a tabloid and focuses a lot on celebrity news and gossip. It must be noted that in all these cases the lines between casual and formal are not so clear as presented here. For instance, CNN will also discuss celebrities and the Daily Mail will also discuss politics, but the more common topics will be the ones with a larger influence on the word embeddings and so for the sake of the paper these categories will be used.
	
The datasets used for the news sites comes from Hermann et al. for their 2015 paper “Teaching Machines to Read and Comprehend”. They gathered CNN articles from 2007 until 2015 and Daily Mail articles from 2010 until 2015. The researchers posted their program for the set of articles. These datasets were then hosted online by Kyunghyun Cho at New York University,\footnote{https://cs.nyu.edu/~kcho/DMQA/} which is the dataset used in this study. For the Reddit data, we used a dataset made and hosted by pushshift.io.\footnote{https://files.pushshift.io/reddit/comments/} This group gathers all reddit posts and comments, with metadata as provided by Reddit, for a given month and then hosts that file on their website. The months used in this study are from September 2019 and January 2014. The reason for having two months is that we wanted the most recent data available (2019) and a dataset from the same time frame as the news datasets (2014). We do not expect there to be a significant difference between the separate reddit datasets.
	
To ensure that each word embedding model was relatively equal in their representation, each model was made from a corpus of 35 million words. This is considered small for a corpus for building a word2vec model, however we were limited by the file limit size on github which was used for sharing data among researchers. Furthermore, the news Reddit data from January 2014 only included 34 422 118 words, so a limit of 35 million words could also ensure all the corpuses were roughly even in size. To ensure that the Reddit data was comprised of comments which was in fact discussion, we removed any comment shorter than 10 words. For data cleanup purposes the first line of each news article was omitted as it included text that was not part of the article and it did not always follow the same format. Table 1 lays out the final relevant statistics for each dataset, where “All xxxx” represents the dataset of all reddit comments from the relevant year and “News xxxx” represents the dataset of reddit comments pertaining to news and politics of the relevant year.

\begin{table}[H]
\begin{center} 
\caption{Dataset Statistics.} 
\label{dataset-stat} 
\vskip 0.12in
\begin{tabular}{llll} 
\hline
Dataset    & \# of    & \# of Total & \# of Unique \\
           &  Entries & Words       & Words        \\
\hline
All 2019   & 820 587  & 35 000 247  & 81 008       \\
All 2014   & 845 925  & 35 000 004  & 72 218       \\
News 2019  & 760 393  & 35 000 039  & 49 897       \\
News 2014  & 664 873  & 34 422 118  & 52 258       \\
CNN        & 54 403   & 35 000 478  & 75 810       \\
Daily Mail & 58 031   & 35 000 381  & 79 159       \\
\hline
\end{tabular} 
\end{center} 
\end{table}

\subsection{Analysis}

TYPE HERE

\section{General Formatting Instructions}

The entire content of a paper (including figures, references, and anything else) can be no longer than six pages in the \textbf{initial submission}. In the \textbf{final submission}, the text of the paper, including an author line, must fit on six pages. Up to one additional page can be used for acknowledgements and references.

The text of the paper should be formatted in two columns with an
overall width of 7 inches (17.8 cm) and length of 9.25 inches (23.5
cm), with 0.25 inches between the columns. Leave two line spaces
between the last author listed and the text of the paper; the text of
the paper (starting with the abstract) should begin no less than 2.75 inches below the top of the
page. The left margin should be 0.75 inches and the top margin should
be 1 inch.  \textbf{The right and bottom margins will depend on
  whether you use U.S. letter or A4 paper, so you must be sure to
  measure the width of the printed text.} Use 10~point Times Roman
with 12~point vertical spacing, unless otherwise specified.

The title should be in 14~point bold font, centered. The title should
be formatted with initial caps (the first letter of content words
capitalized and the rest lower case). In the initial submission, the
phrase ``Anonymous CogSci submission'' should appear below the title,
centered, in 11~point bold font.  In the final submission, each
author's name should appear on a separate line, 11~point bold, and
centered, with the author's email address in parentheses. Under each
author's name list the author's affiliation and postal address in
ordinary 10~point type.

Indent the first line of each paragraph by 1/8~inch (except for the
first paragraph of a new section). Do not add extra vertical space
between paragraphs.


\section{First Level Headings}

First level headings should be in 12~point, initial caps, bold and
centered. Leave one line space above the heading and 1/4~line space
below the heading.


\subsection{Second Level Headings}

Second level headings should be 11~point, initial caps, bold, and
flush left. Leave one line space above the heading and 1/4~line
space below the heading.


\subsubsection{Third Level Headings}

Third level headings should be 10~point, initial caps, bold, and flush
left. Leave one line space above the heading, but no space after the
heading.


\section{Formalities, Footnotes, and Floats}

Use standard APA citation format. Citations within the text should
include the author's last name and year. If the authors' names are
included in the sentence, place only the year in parentheses, as in
\citeA{NewellSimon1972a}, but otherwise place the entire reference in
parentheses with the authors and year separated by a comma
\cite{NewellSimon1972a}. List multiple references alphabetically and
separate them by semicolons
\cite{ChalnickBillman1988a,NewellSimon1972a}. Use the
``et~al.'' construction only after listing all the authors to a
publication in an earlier reference and for citations with four or
more authors.


\subsection{Footnotes}

Indicate footnotes with a number\footnote{Sample of the first
footnote.} in the text. Place the footnotes in 9~point font at the
bottom of the column on which they appear. Precede the footnote block
with a horizontal rule.\footnote{Sample of the second footnote.}


\subsection{Tables}

Number tables consecutively. Place the table number and title (in
10~point) above the table with one line space above the caption and
one line space below it, as in Table~\ref{sample-table}. You may float
tables to the top or bottom of a column, and you may set wide tables across
both columns.

\begin{table}[H]
\begin{center} 
\caption{Sample table title.} 
\label{sample-table} 
\vskip 0.12in
\begin{tabular}{ll} 
\hline
Error type    &  Example \\
\hline
Take smaller        &   63 - 44 = 21 \\
Always borrow~~~~   &   96 - 42 = 34 \\
0 - N = N           &   70 - 47 = 37 \\
0 - N = 0           &   70 - 47 = 30 \\
\hline
\end{tabular} 
\end{center} 
\end{table}


\subsection{Figures}

All artwork must be very dark for purposes of reproduction and should
not be hand drawn. Number figures sequentially, placing the figure
number and caption, in 10~point, after the figure with one line space
above the caption and one line space below it, as in
Figure~\ref{sample-figure}. If necessary, leave extra white space at
the bottom of the page to avoid splitting the figure and figure
caption. You may float figures to the top or bottom of a column, and
you may set wide figures across both columns.

\begin{figure}[H]
\begin{center}
\fbox{CoGNiTiVe ScIeNcE}
\end{center}
\caption{This is a figure.} 
\label{sample-figure}
\end{figure}


\section{Acknowledgments}

In the \textbf{initial submission}, please \textbf{do not include
  acknowledgements}, to preserve anonymity.  In the \textbf{final submission},
place acknowledgments (including funding information) in a section \textbf{at
the end of the paper}.


\section{References Instructions}

Follow the APA Publication Manual for citation format, both within the
text and in the reference list, with the following exceptions: (a) do
not cite the page numbers of any book, including chapters in edited
volumes; (b) use the same format for unpublished references as for
published ones. Alphabetize references by the surnames of the authors,
with single author entries preceding multiple author entries. Order
references by the same authors by the year of publication, with the
earliest first.

Use a first level section heading, ``{\bf References}'', as shown
below. Use a hanging indent style, with the first line of the
reference flush against the left margin and subsequent lines indented
by 1/8~inch. Below are example references for a conference paper, book
chapter, journal article, dissertation, book, technical report, and
edited volume, respectively.

\nocite{ChalnickBillman1988a}
\nocite{Feigenbaum1963a}
\nocite{Hill1983a}
\nocite{OhlssonLangley1985a}
% \nocite{Lewis1978a}
\nocite{Matlock2001}
\nocite{NewellSimon1972a}
\nocite{ShragerLangley1990a}


\bibliographystyle{apacite}

\setlength{\bibleftmargin}{.125in}
\setlength{\bibindent}{-\bibleftmargin}

\bibliography{CogSci_Template}


\end{document}